\documentclass{article}

\usepackage[english]{babel}
\usepackage{amsthm}
\usepackage{amsmath}
\usepackage{amssymb}
\usepackage{xcolor}
\usepackage{graphicx}
\usepackage{geometry}
\usepackage{float}

\geometry{
    a4paper,
    total={170mm,257mm},
    left=20mm,
    top=20mm,
}
\usepackage{hyperref}
% \hypersetup{
%     colorlinks=true,
%     linkcolor=blue,
%     filecolor=magenta,      
%     urlcolor=cyan,
%     pdftitle={Overleaf Example},
%     pdfpagemode=FullScreen,
% }

\usepackage{cleveref}


\graphicspath{{./images/}}

\theoremstyle{definition}
\newtheorem{definition}{Definition}[section]

\newtheorem{theorem}{Theorem}[section]
\newtheorem{lemma}[theorem]{Lemma}
\newtheorem{proposition}[theorem]{Proposition}
\newcommand{\blue}[1]{\textcolor{blue}{#1}}

\title{Special Properties of Gradient Descent with Large Learning Rates}
\author{Denis Grachev}

\begin{document}
\maketitle

\begin{abstract}
This article is an overview of the work \cite{mohtashami2023special}
with some additional experiments. 
It has been widely observed that usage of larger learning 
rate in stochastic gradient descent often results into better models.
However theoretical reasons for this phenomena are not well understood yet.
Previous studies linked it to stochastic noise in SGD. 
The work focuses on theoretical proof of escaping sharp minima 
for some special class of functions. 
Also it is shown that for certain starting points and loss functions
GD with large learning rate has different trajectory 
and may lead to convergence to another minima, 
which is likely to be more robust. 

\end{abstract}

% \section{Introduction}

\section{Main results}
Usage of large learning rate is often explained by intuition of 
escaping local minima. 
In this work a class $C_l$ of functions is constructed 
which have at least two minima ($x^\dagger$ and $x_\ast$). 
Randomly initialized GD with large learning rate converges 
to $x_\ast$ almost surely, but with smaller learning rate 
there is strictly positive probability of converging to $x^\dagger$.

\section{Theoretical Analysis}
For analysis full-batch gradient descent with random initialization was taken:
$$ f_* := \min_{x \in \mathbb{R}^d} f(x).$$
$f$ is supposed to be $L$-smooth over regions of the landscape 
so that the gradient does not change too sharply. 
Also we would require sharpness of some regions of $f$ 
around local minima using 
$\mu$-one-point-strongly-convexity (OPSC) with respect to $x_\ast$ over $M$.

\begin{definition}[$L$-smoothness]
    A function $f: \mathbb{R}^d \rightarrow \mathbb{R}$ 
    is $L$-smooth if $f$ is differentiable and 
    exists $L: \| \nabla f(x) - \nabla f(y)\| \leq L \| x - y \|, \: \forall x, y \in \mathbb{R}^d$ 
\end{definition}

\begin{definition}[$\mu$-one-point-strongly-convex (OPSC) with respect to $x_\ast$ over $M$]
    A function $f: \mathbb{R}^d \rightarrow \mathbb{R}$ is 
    $\mu$-one-point-strongly-convex (OPSC) with respect to $x_\ast$ over $M$ 
    if it is differentiable and 
    $$\exists \mu > 0: \langle \nabla f(x), x - x_\ast \rangle \geq \mu \| x - x_\ast\|^2, \: \forall x \in M.$$
\end{definition}

\begin{lemma}\label{lma1}
    Let $f$ be a function that is 
    $L_{\mathrm{global}}$ -smooth with a global minimum $x_\ast$. 
    Assume there exists a local minimum $x^\dagger$ around which
    \begin{itemize}
        \item $f$ is $\mu^\dagger$-OPSC woth respect to 
        $x^\dagger$ over a set $M$ that contains $x^\dagger$ with diameter $r$.
        
        \item Let $P(M)$ be a ball around $x^\dagger$ with radius 
        $r_P$ excluding points M. $f$ is 
        $L < L_{\mathrm{global}}$ -smooth in $P(M)$ and 
        $\mu_\ast$-OPSC with respect to $x_\ast$, 
        such as $\mu^\dagger > \frac{2L^2}{\mu_\ast}$. 
        $r_P$ depends on $r, \gamma, L_{\mathrm{global}}$.

        \item $\| x_\ast - x^\dagger \| > \tau$, where $\tau$ depends on $\mu_\ast, r, \gamma$. 
    \end{itemize}

        Then using learning rate $\frac{2}{\mu^\dagger} < \gamma < \frac{\mu_\ast}{L^2}$
        GD escape $M$ and reach a point closer to $x_\ast$ than 
        $\| x^\dagger - x_\ast\| - r$ almost surely. 

\end{lemma}
\begin{proof}
    Proof is presented in \cite{mohtashami2023special} in appendix D.
\end{proof}


Figure \cref{fig:func} illustrates regions and points defined in \cref{lma1} for a simple function.

\begin{figure}[H]
    \caption{Illustation of regions from \cref{lma1}}
    \label{fig:func}
    \centering
    \includegraphics[scale=0.2]{lemma1}
\end{figure}

\begin{theorem}\label{thm1}
    Let $C_l$ be the set of functions such as $f$ is 
    $L$-smooth and $\mu_\ast$-OPSC with respect to the 
    global minima $x_\ast$ except n a region $M$ that 
    contains local minima $x^\dagger$ and satisfies \cref{lma1}.
    \begin{itemize}
        \item Gradient descent initialized randomly inside $M$ 
        with learning rate $\gamma < \frac{\mu^\dagger}{L^2_{\mathrm{global}}}$
        converges to $x^\dagger$ almost surely.
        
        \item Gradient descent initialized randomly in 
        arbitrary set $W: \mathcal{L}(W) > 0$ 
        with learning rate $\frac{2}{\mu^\dagger} < \gamma \leq \frac{\mu_\ast}{L^2}$
        converges to $x_\ast$ almost surely.

    \end{itemize} 

\end{theorem}
\begin{proof}
    Proof is presented in \cite{mohtashami2023special} in appendix E.
\end{proof}

\begin{lemma}\label{lma2}
    Consider gradient descent initialized randomly in set $W$ 
    with learning rate $\gamma \leq \frac{1}{2L}$.
    Let $X \subset \mathbb{R}^d$ arbitrary set of points in 
    the landscape, $f$ is $L$-smooth over 
    $\mathbb{R}^d \setminus X$. Probability of encountering 
    any point of $X$ in first $T$ steps of gradient descent is 
    at most $2 ^ {(T + 1)d} \frac{\mathcal{L}(X)}{\mathcal{L}(W)}$.
\end{lemma}
\begin{proof}
    Proof is presented in \cite{mohtashami2023special} in appendix B.
\end{proof}

\begin{theorem}\label{thm2}
    Let $X$ be an arbitrary set of points, 
    $f$ is $\mu_\ast$-OPSC with respect to a minima 
    $x_\ast \notin X$ over $\mathbb{R}^d \setminus X$.
    Let $c_X := \inf \left\{ \| x - x_\ast \| \:|\: x \in X  \right\}$
    and $r_W := \sup \left\{ \| x - x_\ast \| \:|\: x \in W \right\}$.
    The probability of not encountering any point of $X$ during 
    gradient descent with learning rate $\gamma \leq \frac{\mu_\ast}{L^2}$
    is at least $1 - \frac{r_W}{c_X}^{\frac{-d}{\log_2(1 - \gamma \mu_\ast)}} \frac{\mathcal{L}(X)}{\mathcal{L}(W)} 2^d$
    if $c_X \leq r_W$ and $1$ otherwise.
\end{theorem}

\begin{proof}
    Proof is presented in \cite{mohtashami2023special} in appendix C.
\end{proof}

% Take the case of SGD 
% $$ 
% x_{t + 1} := x_t - \gamma \left( 
%     \nabla f(x_t) + \xi_t
% \right),
% $$
% where $\xi_t$ independent and identically distributed
% from $\mathrm{Uniform}(-\sigma, \sigma)$. 

% \begin{proposition}
%     Consider SGD on the function from figure \ref{fig:func}, 
%     starting close to $x^\dagger$.
%     If the learning rate is sufficiently small the iterations
%     will never converge to $x_\ast$ nor to a small region around it, 
%     regardless of the magnitude of the noise. 
%     If the learning rate is large enough and stochastic noise
%     satisfies certain bounds, SGD will converge to $x_\ast$ 
%     from any starting point.
% \end{proposition}

% \begin{proof}
    
% \end{proof}

\clearpage 

\section{Experiments}
\subsection*{1D Example}
Study different GD behaviour depending on starting point and learning rate 
on the function $f$. 
$$
f(x):= \begin{cases}
    -1600(x-2.5)^5-2000(x-2.5)^4+800(x-2.5)^3+1020(x-2.5)^2 & 2 \leq x \leq 3 \\ 
    1411.2 \times\left(1-10^4(x-8.4)\right) & 8.4 \leq x \leq 8.40001 \\ 
    0 & 8.40001 \leq x \leq 8.59999, \\ 
    1479.2 \times\left(10^4(x-8.6)+1\right) & 8.59999 \leq x \leq 8.6, \\ 
    20 x^2 & \textit{ otherwise }
\end{cases}
$$

\begin{figure}[!htb]
    \caption{Visual comparison of GD with different learning rates.}
    \minipage{0.33\textwidth}
      \includegraphics[width=\linewidth]{converge_to_sharp.png}
      \caption{GD converged to the sharpest minima.}
    \endminipage\hfill
    \minipage{0.32\textwidth}
      \includegraphics[width=\linewidth]{converge_to_half_sharp.png}
      \caption{GD converged to the sharp minima.}
    \endminipage\hfill
    \minipage{0.32\textwidth}%
      \includegraphics[width=\linewidth]{converge_to_flat.png}
      \caption{GD converged to the flat minima.}
    \endminipage
\end{figure}

\begin{figure}[!htb]
    \begin{center}
        \caption{
            Types of convegence of GD depending on startpoint and learning rate. \\
            Empty spaces represent no convergence in 1000 iterations.
        }
        \includegraphics[scale=0.7]{1d_GD_results.png}        
    \end{center}
\end{figure}

From the experiments we can see that usage of larger learning rates
helps to escape sharp minima and converge to a flatter one.

\clearpage

\subsection{2D Example}
Study different GD behaviour depending on learning rate on
the function $f$

$$
f(x, y):=x^2+y^2-200 \mathrm{ReLU}(|x|-1) \mathrm{ReLU}(|y|-1) \mathrm{ReLU}(2-|x|) \mathrm{ReLU}(2-|x|)
$$


\begin{figure}[h]
    \caption{Visual comparison of GD with different learning rates.}
    \minipage{0.48\textwidth}
      \includegraphics[width=\linewidth]{2d_example_flat.png}
      \caption{GD converged to flat minima.}
    \endminipage\hfill
    \minipage{0.48\textwidth}
      \includegraphics[width=\linewidth]{2d_example_sharp.png}
      \caption{GD converged to sharp minima.}
    \endminipage
\end{figure}

\begin{figure}[H]
    \begin{center}
        \caption{Share of GDs that obtained flat minima if starting point is random $3 \leq x, y \leq 4$.}
        \includegraphics[scale=0.3]{2d_example_lr_vs_minima.png}        
    \end{center}
\end{figure}

From these toy examples we can see that GD with larger learning rate 
indeed converge to flatter minima. 

\subsection{MNIST example}
For the experiment with neural networks MNIST dataset was taken. 
The architecture of the NN consisted of 3 linear layers with $\mathrm{ReLU}$ 
activation layers each and cross entropy loss function.
During the experiment 3 random initial points were chosen and from each starting point 
3 GDs with different learning rates were performed. 
Weights of the first layer were saved during training and 
then were reduced to 2 dimensional space using PCA. 
PCA was performed in 2 ways.
The first way is all starting points together, 
the second one is each starting point separately from the others. 

\begin{figure}[H]
    \begin{center}
        \caption{Trajectories of all starting points}
        \includegraphics[scale=0.3]{pca_all_strategies.png}        
    \end{center}
\end{figure}

From this picture it seems like the trajectory does not depend much 
on a learning rate and GDs from same point follow similar path. 
But if we look at trajectories when PCA performed for each starting point independently 
we can see that the paths are different. 
Also some symmetry across different starting points can be noted. 

\begin{figure}[!htb]
    \caption{Trajectories of each starting point.}
    \minipage{0.33\textwidth}
      \includegraphics[width=\linewidth]{trajectory_a.png}
      \caption{GD for starting point A.}
    \endminipage\hfill
    \minipage{0.33\textwidth}
      \includegraphics[width=\linewidth]{trajectory_b.png}
      \caption{GD for starting point B.}
    \endminipage\hfill
    \minipage{0.33\textwidth}
      \includegraphics[width=\linewidth]{trajectory_c.png}
      \caption{GD for starting point C.}
    \endminipage
\end{figure}

All the obtained minimas showed similar performance on train and test validation.

\section{Conclusion}
It was shown that for certain class of functions and with weak assumptions on 
starting point GD with small learning rate almost surely converges to sharp minima, 
while GS with larger learning rate almost surely converges to flat minima. 
This behaviour was shown on one and two dimensional cases. 
Similar behaviour was shown for neural networks. 
For different starting points GDs with different learning rate converged to different minimas. 

\section{Future work}
For more illustrative experiments a pretrained and more complex neural network might be taken. 
It is initially near some minima, and different learning rate can lead to converging to it, 
or escaping from it and search for a flatter minima. 
In this case obtained trajectories might have more notable difference. 
Also for more complex architectures flatter minima might be more robust 
and performe better on testing set. 

\bibliographystyle{alpha}
\bibliography{sample}

\end{document}
